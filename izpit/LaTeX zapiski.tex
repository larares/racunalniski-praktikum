\documentclass{article} 

% documentclass ima v [] nastavitve in v {} tip dokumenta (v [] smo dali kot primer a4paper in velikost pisave - npr [a4paper, 10pt]) -> \documentclass[a4paper]{article}

%alt+z
% pogosti tipi dokumentov:
%   article - članek
%   amsart - članki za revije American Mathematical Society
%   beamer - prosojnice za predavanje
%   book - knjiga
%   letter - pismo

% PREAMBULA = ne pišemo nastavitev ampak samo ukaze in nastavitve, ki jih določamo s pomočjo različnih paketov; \usepackage[dodatne nastavitve]{ime paketa} 

\usepackage[utf8]{inputenc} 
% vhodna datoteka je kodirana v sistemu UTF-8

\usepackage[T1]{fontenc}      
% izhodna datoteka je kodirana v sistemu T1


% PAKETI ZA SLOVENŠČINO:
\usepackage[slovene]{babel} 
% za uporabljanje slovenščine in izpisovanje šumnikov
\usepackage{lmodern}
% to je dodaten, načeloma neobvezen paket, ki olepša slovensko pisavo 
% Ostale pogoste izbire pisav:
% \usepackage{times}            % Times New Roman
% \usepackage{palatino}         % Palatino
% \usepackage{concrete}         % Pisava, ki jo je uporabil Donald Knuth v knjigi "Concrete mathematics"


% OSTALI PAKETI:
\usepackage{hyperref}
% paket za povezave znotraj dokumenta in hiper - povezave
\usepackage{graphicx} 
% za vstavljanje slik
\usepackage{booktabs} 
% za lepše tabele
\usepackage{multirow} 
% za vnose v tabeli čez več vrstic
\usepackage{siunitx}  
% za poravnavo na decimalni vejici v tabeli
\usepackage{listings} 
% za prikaz programske kode
\usepackage{pgfplots} 
% za risanje grafa funkcije
\pgfplotsset{compat=1.18}


%PAKETI ZA MATEMATIKO:
\usepackage{amsmath}  
% razna okolja za poravnane enačbe ipd.
\usepackage{amsthm}   
% definicije okolij za izreke, definicije, ..., ukaza theoremstyle, newtheorem
\usepackage{amssymb} 
% dodatni matematični simboli
\usepackage{xypic}    
% paket za diagrame

\newcommand{\RR}{\mathbb{R}} % realna števila, \mathbb{..} je iz amssymb
\newcommand{\NN}{\mathbb{N}} % naravna števila
\newcommand{\pair}[1]{\langle #1 \rangle}
\newcommand{\zap}[2]{#1_1, \ldots, #1_{#2}}

{\theoremstyle{definition}
\newtheorem{definicija}{Definicija}
}
%uporaba: \begin{definicija} -> zapišeš definicijo oz karkoli že moaš zapisati
{\theoremstyle{plain}
\newtheorem{izrek}{Izrek}
}
% {\theoremstyle{proof}
% \newtheorem{dokaz}{Dokaz}
% }
\newcommand{\pojem}[1]{\emph{\color{purple}#1}} %neki smo s tem delali v magišnem kvadratu


% KONEC PREAMBULE


% Definicija okolij izrek, posledica
%{\theoremstyle{plain}
%\newtheorem{izrek}{Izrek}[section]
%\newtheorem{posledica}[izrek]{Posledica}
%}

% Definicija okoli za definicije in vaje
%{\theoremstyle{definition}
%\newtheorem{definicija}[izrek]{Definicija}
%\newtheorem{vaja}[izrek]{Vaja}
%}
%to tukej ne bo delal zato ker sem že definirala definicijo nekje vmes


% VSEBINA DOKUMENTA - se vedno začne z:
\begin{document}
% in kasneje konča z \end{document}% NASLOV IN AVTOR

\title{Zapiski za {\LaTeX}}
% naslov

\author{Lara Reš}
% in avtor


\date{}
% lahko nastavimo še datum, ki ga zapišemo v zavite oklepaje, če pa želimo, da se datum ne izpiše, potem pustimo prazno znotraj oklepajev


\maketitle 
% je ukaz, s katerim dejansko naredimo naslov


% OKOLJA

\begin{abstract}
    To je okolje za povzetek, napišemo kratek opis vsebine članka
\end{abstract}

\begin{quote}
Daljši navedek
\end{quote}

\begin{center}
  centrirano besedilo
\end{center}

\begin{flushleft}
  poravnano v levo
\end{flushleft}

\begin{flushright}
  poravnano v desno
\end{flushright}


%MATEMATIČNI IZRAZI
\begin{itemize}
  \item $a^2 + b^2 = c^2$
  \item \(a^2 + b^2 = c^2\)
  \item \[
    a^2 + b^2 = c^2.
  \]
  %matematični izraz na sredini vrstice
  \item $$
    a^2 + b^2 = c^2.
  $$
  \item \(a^2 + b^2 = c^2\) %matematični izraz v vrstici
  \item okolje %\math za vrstični prikaz
  \item okolje %\displaymath za prikazni način
\end{itemize}

Ulomke zapišemo tako: $\frac{22}{7}$
$\displaystyle \frac{2}{3}$
$2/3$

\rule[12pt]{10pt}{10pt}
%\rule[⟨višina⟩]{⟨dolžina⟩}{⟨širina⟩} - nariše pravokotnik

pika za množenje: $\cdot$

% Pravila za pisanje izrazov v prikaznem načinu:
% 1. Ne delamo praznih vrstic pred in za izrazom, razen se s formulo konča stavek.
% 2. Izraz mora biti del stavka, ne sme stati kar prepuščen sam sebi.
% 3. Če se z izrazom konča stavek ali odvisni stavek, zapišemo tudi ločilo.
Pravilno: za nenegativna števila $x_1, \ldots, x_n$ velja
%
\[
  \frac{x_1 + \cdots + x_n}{n} \leq
  \sqrt{\frac{x_1^2 + \cdots + x_n^2}{n}},
\]
%
kjer enakost velja natanko tedaj, ko $x_1 = x_2 = \cdots = x_n$.

Če želimo vstaviti v formulo besedilo, to naredimo s \verb|\text{...}|:
%
\begin{equation*}
  \mathcal{P} = \{ n \in \mathbb{N} \mid \text{$n$ je praštevilo} \}.
\end{equation*}
%
Omenimo še okolje \texttt{cases}, s katerim obravnavamo primere:
%
\[
  f(x) =
  \begin{cases}
    -1 & \text{če $x < 1$,} \\
     x & \text{če $-1 \leq x \leq 1$,} \\
     1 & \text{če $1 < x$.}
  \end{cases}
\]

%makroje smo definirali v preambuli
Ali je res, da za vsak $n \in \NN$ in $a \in \RR$ obstaja natanko en $b \in \RR$, da je $b^n = a$?

% Ukaz \par že obstaja, zato ne moremo definirati makroja \par

Ali se urejeni pari pišejo $(x, y)$ ali $\pair{x, y}$? Kdo bi vedel. Je pa tako, da se
splača uporabiti makro, ker lahko kasneje še spreminjamo njegovo definicijo.

Ne počnimo neumnosti z makoroji: če seštejemo zaporedji $\zap{a}{n}$ in $\zap{b}{n}$,
dobimo zaporedje $\zap{a + b}{n}$.

% Pozor, indekse je treba pisati v {...}, sicer se zgodijo čudne stvari.
% V definiciji \zap pobrišite {...} okoli #2 in poskusite:
Kaj pa dobimo, če napišemo $\zap{x}{n + m}$?

%Oklepaji

Poznamo več vrst oklepajev: $($ in $)$, $[$ in $]$, $\{$ in $\}$, $\langle$ in $\rangle$.

Še posebej opozorimo na $\langle$ in $\rangle$. Prav se piše $\langle x, y\rangle$ in ne
$<x, y>$, ker {\LaTeX} obravnava simbola \texttt{<} in \texttt{>} kot relaciji `večje' in
`manjše'.

Včasih so oklepaji premahjni, denimo
%
\[
  (\frac{a}{b})^n = \frac{a^n}{b^n}.
\]
%
V takih primerih pred uklepaj napišemo \verb|\left| in pred zaklepaj \verb|\right|:
%
\[
  \left(\frac{a}{b}\right)^n = \frac{a^n}{b^n}.
\]
%
Velikost oklepajev lahko nadzorujemo tudi ročno z ukazi \verb|\big|, \verb|\Big|,
\verb|\bigg|, \verb|\Bigg|:
%
\[
  \Bigg( \bigg( \Big( \big( ( ) \big) \Big) \bigg) \Bigg)
\]

Paket \texttt{amsmath} vsebuje okolja za prikaz raznih enačb. Če uporabimo okolje brez
zvezdice, na primer \texttt{equation}, dobimo oštevilčeno enačbo, okolje z zvezdico, na
primer \texttt{equation*}, pa nam da neoštevilčeno enačbo. Vedno oštevilčimo samo tiste
enačbe, na katere se tudi sklicujemo.

Običajne enačbe naredimo z okoljem \texttt{equation},
\begin{equation*}
  x^2 + y^2 = 1
\end{equation*}
%
ali
%
\begin{equation}
  \label{eq:1}
  x^2 + y^2 = 1
\end{equation}

Z okoljem \texttt{gather} stavimo izraze enega pod drugega:
%
\begin{gather*}
  \log 2 = 1 - \frac{1}{2} + \frac{1}{3} - \frac{1}{4} + \cdots, \\
  \frac{2}{\frac{1}{x} + \frac{1}{y}} \leq \sqrt{x y} \leq \frac{x + y}{2}, \\
  \sum_{k = 1}^\infty \frac{1}{k^2} = \frac{\pi^2}{6}.
\end{gather*}

Z okoljem \texttt{multtline} zapišemo daljšo izpeljavo čez več vrstic. Prva vrstica je
poravnana levo, zadnja desno in vsem vmesne sredinsko:
%
\begin{multline*}
  \sum_{i=1}^n x_i^2 \cdot \sum_{i=1}^n y_i^2
  - \left( \sum_{i=1}^n x_i y_i \right)^2 = \\
  \frac{1}{2} \cdot \sum_{i=1}^n x_i^2 \cdot \sum_{i=1}^n y_i^2
  +
  \frac{1}{2} \cdot \sum_{i=1}^n x_i^2 \cdot \sum_{i=1}^n y_i^2
  -
  \sum_{i=1}^n x_i y_i \cdot \sum_{j=1}^n x_j y_j = \\
  \frac{1}{2} \cdot \sum_{i,j=1}^2 x_i^2 y_j^2
  +
  \frac{1}{2} \cdot \sum_{i,j=1}^n x_j^2 y_i^2
  -
  \sum_{i,j=1}^n x_i y_j x_j y_i = \\
  \sum_{i,j=1}^n \frac{1}{2}\,
  \left(
    x_i^2 y_j^2 + x_j^2 x_i^2 - 2 x_i y_j x_j y_i
  \right)
  =
  \sum_{i,j=1}^n \frac{1}{2}\,
  \left(
    x_i y_j - x_j y_i
  \right)^2
  \geq 0.
\end{multline*}

Z okoljem \texttt{align} lahko poravnamo vrstice na določen znaku. Mesto, kjer morajo biti
vrstice poravnane, označimo z znakom \texttt{\&}, prehod v novo vrsto označimo z \verb|\\|:
%
\begin{align*}
  (x + y)^2 - (x - y)^2
  &= (x^2 + 2 x y + y^2) - (x^2 - 2 x y + y^2) \\
  &= x^2 + 2 x y + y^2 - x^2 + 2 x y - y^2 \\
  &= 2 x y + 2 x y \\
  &= 4 x y.
\end{align*}

\begin{align*}
  (x + y)^2 - (x - y)^2
  &= (x^2 + 2 x y + y^2) - (x^2 - 2 x y + y^2) \\
\intertext{in zato}
  &= x^2 + 2 x y + y^2 - x^2 + 2 x y - y^2 \\
  &= 2 x y + 2 x y \\
  &= 4 x y.
\end{align*}

Z okoljem \texttt{align} lahko poravnamo več stolpcev, ki jih ločimo z znakom \texttt{\&}:
%
\begin{align*}
  3 + 5 &= 8
  &
  2 + 2 &= 4
  &
  1 + 1 &= 2
  \\
  3 + 7 &= 10
  &
  4 + 1 &= 5
  &
  2 + 3 &= 5
\end{align*}

Matriko naredimo z okoljem \texttt{matrix}:\footnote{Tu imamo izjemo, ko je pisanje ločila
  na koneč izraza nesmiselno.}
%
\begin{equation*}
  \left[
  \begin{matrix}
    x_{1,1} & x_{1,2} & \cdots & x_{1,n} \\
    x_{2,1} & x_{2,2} & \cdots & x_{2,n} \\
    \vdots  & \vdots  & \ddots & \vdots  \\
    x_{n,1} & x_{n,2} & \cdots & x_{n,n}
  \end{matrix}
  \right]
\end{equation*}
%
Oklepaje postavimo okoli matrike z \verb|\left| in \verb|\right|, da so pravilne
velikosti.

\begin{izrek}
  \label{izrek:max-zvezna}
  Vsaka zvezna funkcija na zaprtem intervalu doseže maksimum.
\end{izrek}

\begin{proof}
  Lorem ipsum dolor sit amet, consectetur adipiscing elit. Suspendisse aliquet arcu sit
  amet augue consequat efficitur. Nam non diam congue, porttitor nisl nec, faucibus ex.
  Fusce arcu ligula, molestie sit amet ligula sed, finibus sagittis felis. Nulla facilisi.
  Suspendisse potenti
  %
  \[
    f(x) = \int_0^x f'(t) \, d t,
  \]
  %
  donec ultrices malesuada bibendum. Quisque ac rutrum orci. Aliquam
  laoreet euismod nulla fermentum fringilla. Fusce bibendum dui enim, sed luctus diam
  lacinia sit amet. Fusce suscipit sodales vulputate. Suspendisse euismod ante est, ut
  fermentum mi consequat vitae. Sed vehicula, odio quis aliquam tincidunt, massa dolor
  tristique ligula, tempus egestas libero sem ac leo.
\end{proof}

%\begin{posledica}
%  Parbola ima maksimum na zaprtem intervalu.
%\end{posledica}

% Tu oprabimo \qedhere, da se znak kar v formlo.
\begin{proof}
  To sledi iz računa
  %
  \begin{align*}
    (x + y)^2 - (x - y)^2
    &= (x^2 + 2 x y + y^2) - (x^2 - 2 x y + y^2) \\
    &= x^2 + 2 x y + y^2 - x^2 + 2 x y - y^2 \\
    &= 2 x y + 2 x y \\
    &= 4 x y. \qedhere
  \end{align*}
\end{proof}

% V definiciji moramo vedno poudariti pojem, ki ga definiramo. To naredimo z \emph{...}.
\begin{definicija}
  \emph{Praštevilo} je tako naravno število $n$, večje od $1$, ki ni deljivo z nobenim
  naravnih šetvilom.
\end{definicija}

%\begin{vaja}
%  Na pamet izračunajte $23 \times 117$.
%\end{vaja}


% Članek lahko hierarhično razdelimo na razdelke (\section), podrazdelke (\subsection), podpodrazdelke (\subsubsection) in podpodpodrazdelke (\paragraph). V večini primerov potrebujemo samo razdelke in podarazdelke.

\section{Uporaba razdelkov}

\subsection{Uporaba podrazdelkov}

\subsubsection{Uporaba podpodrazdelkov}


%PISAVE

\emph{napiše ležeče besedilo}

\textbf{krepko}

\emph{\textbf{oboje skupaj}}.

Kaj se zgodi, če \emph{uporabimo poudarjeno \emph{znotraj} poudarjenega besedila}?
Seveda \textbf{lahko lahko nastavimo tudi \textnormal{običajno} pisavo}.
Besedilo lahko tudi \underline{podčrtamo}, vendar tega ne priporočamo, \underline{ker} \underline{je grdo}.
Pišemo lahko tudi v \textsf{sans-serifni pisavi} ali pa \textsc{z malimi velikimi črkami},
denimo \textsc{Python}. Ležeča pisava \textsl{ni ista reč kot} \emph{poudarjena pisava}.
Včasih uporabimo tudi \texttt{pisavo fiksne širine}, v kateri so vsi znaki enako široki.

% Ukazi s katerimi nastavimo velikost pisave, a tega raje ne počnite na roke
\begin{center}
{\Huge Zdravljica} \\
{\Huge Živé naj vsi naródi,} \\
{\huge ki hrepené dočakat dan,} \\
{\LARGE ko, koder sonce hodi,} \\
{\Large prepir iz svéta bo pregnan,} \\
{\large ko rojak} \\
{\normalsize prost bo vsak,} \\
{\footnotesize ne vrag, le sosed bo mejak!} \\
{\scriptsize Kdor ne skače ni Slovenc!} \\
\end{center}


%NAŠTEVANJE

Neoštevilčeno naštevanje:
%
\begin{itemize}

\item stvar

\item šeena stvar
\end{itemize}

Oštevilčeno naštevanje:
%
\begin{enumerate}

\item Prvi

\item drugi

\end{enumerate}

Vgnezdeno naštevanje:
%
\begin{enumerate}
\item bla
\item blabla
  \begin{itemize}
  \item bla
  \item blabla
    \begin{itemize}
    \item blabla
    \item blablabla
    \item blablablabla
    \end{itemize}
  \item bla
    \begin{enumerate}
    \item blabla
    \item blablalbal
    \item blabla
    \end{enumerate}
  \end{itemize}
\end{enumerate}


%PRESLEDKI
~ pomeni kratek presledek - npr.: akad.~prof.~dr.~France namesto akad. prof. dr. France


%VKLJUČEVANJE SLIKE
\begin{center}
%  \includegraphics{slika.jpg}
\end{center}
%center nam postavi sliko na sredino, slika je zakomentirana zato, da se v pdfju ne pokažeta dve


%OBLIKOVANJE SLIK
\begin{figure}
  \centering
  \includegraphics[width=0.5\textwidth]{slika.jpg}
  \caption{Prvi zadetek na Google za ``very good''}
\end{figure}


%TABELE
Primer magičnega kvadrata reda 3 je prikazan v tabeli \ref{table:mag3}.
\begin{table}[!ht]
   \centering % da je poravnanan na sredino
   \large
   \caption{Magični kvadrat reda 3} %napis nad tabelo
   \label{table:mag3} %oznaka - je oznaka za sklicevanje
\begin{tabular}{|c|l|r|} %kje se poravna vsak stolpec, toliko kot je teh različnih prostorčkov, toliko je stolpcev, da napišeš poševne črte je alt+w |
   \hline
   8 & 1 & 6 \\\hline %celice ločuješ z &
   3 & 5 & 7 \\\hline
   4 & 9 & 2 \\\hline
\end{tabular}   
\end{table}

V tabeli~\ref{tab:volitve-vanilla} vidimo rezultate volitev, uporabili so navaden {\LaTeX}.
V tabeli~\ref{tab:volitve-booktabs} vidimo rezultate volitev, uporabili smo paket \texttt{booktabs}.
V tabeli~\ref{tab:volitve-align} vidimo rezultate volitev s poravnanimi decimalnimi pikami in vejicami.

% h = here (tukaj)
% t = top (na vrhu strani)
% p = page (na svoji strani)
\begin{table}[htp]
  \centering
  \begin{tabular}{|l|r|r|}
  \hline
  \textbf{Kandidat/Kandidatka}        & \textbf{Odstotek} & \textbf{Število glasov} \\ \hline
  Borut Pahor                & 47,07\%  & 348.938 \\ \hline
  Marjan Šarec               & 24,96\%  & 185.042 \\ \hline
  Romana Tomc                & 13,74\%  & 101.845 \\ \hline
  Ljudmila Novak             & 7,16\%   & 53.049 \\ \hline
  Andrej Šiško               & 2,22\%   & 16.463 \\ \hline
  Boris Popovič              & 1,79\%   & 13.277 \\ \hline
  dr.\ Maja Makovec Brenčič  & 1,72\%   & 12.734 \\ \hline
  Suzana Lara Krause         & 0,77\%   & 5.718 \\ \hline
  Angela (Angelca) Likovič   & 0,58\%   & 4.273 \\ \hline
  \end{tabular}
  \caption{Rezultati predsedniških volitev, kot bi jih prikazali z grdo razpredelnico, ki ima preveč črt.}
  \label{tab:volitve-vanilla}
\end{table}

\begin{table}[htb]
  \centering
  \begin{tabular}{lrr}
  \toprule
  \textbf{Kandidat/Kandidatka}        & \textbf{Odstotek} & \textbf{Število glasov} \\ \midrule
  Borut Pahor                & 47,07\%  & 348.938 \\
  Marjan Šarec               & 24,96\%  & 185.042 \\
  Romana Tomc                & 13,74\%  & 101.845 \\
  Ljudmila Novak             & 7,16\%   & 53.049 \\
  Andrej Šiško               & 2,22\%   & 16.463 \\
  Boris Popovič              & 1,79\%   & 13.277 \\
  dr.\ Maja Makovec Brenčič  & 1,72\%   & 12.734 \\
  Suzana Lara Krause         & 0,77\%   & 5.718 \\
  Angela (Angelca) Likovič   & 0,58\%   & 4.273 \\
  \bottomrule
  \end{tabular}
  \caption{Rezultati predsedniških volitev s paketom \texttt{booktabs}}
  \label{tab:volitve-booktabs}
\end{table}

% V tej tabeli smo števila zapisali 
\begin{table}[htb]
  \centering
  \begin{tabular}{lSS[group-minimum-digits=3]}
  \toprule
  \textbf{Kandidat/Kandidatka}        & \textbf{Odstotek} & \textbf{Število glasov} \\ \midrule
  Borut Pahor                & 47.07\%  & 348938 \\
  Marjan Šarec               & 24.96\%  & 185042 \\
  Romana Tomc                & 13.74\%  & 101845 \\
  Ljudmila Novak             & 7.16\%   & 53049 \\
  Andrej Šiško               & 2.22\%   & 16463 \\
  Boris Popovič              & 1.79\%   & 13277 \\
  dr.\ Maja Makovec Brenčič  & 1.72\%   & 12734 \\
  Suzana Lara Krause         & 0.77\%   & 5718 \\
  Angela (Angelca) Likovič   & 0.58\%   & 4273 \\
  \bottomrule
  \end{tabular}
  \caption{Rezultati predsedniških volitev, s poravnanimi decimalnimi vejicami in pikami}
  \label{tab:volitve-align}
\end{table}

%če se npr druga celica v prvi vrstici razteza čez 5 stolpcev: \multicolumn{5}{|c|}{točna vrednost} in lahko izpuščaš &
%vsako vrstico zaključiš z \\ in \hline

%VKLJUČEVANJE SPLETNIH NASLOVOV
\begin{itemize}
   \item \url{http://mathworld.wolfram.com/MagicSquare.html}
   \item \url{http://en.wikipedia.org/wiki/Magic_square}
\end{itemize}


%UKAZI

\begin{center}
  \verb|\imeUkaza{...}|
\end{center}
%
ali
%
\begin{center}
  \verb|\imeUkaza{...}{...}{...}|
\end{center}
%
če ukaz sprejme več argumentov. Uporabnik lahko definira svoje ukaze, a o tem kasneje.
Ukazi brez argumenta ``pojejo'' presledek, zato jih pišemo v zavite oklepaje:
%
\begin{itemize}
\item \LaTeX je dokumentni sistem.
\item {\LaTeX} je dokumentni sistem.
\end{itemize}


%GIT IGNORE
% v datoteko .gitignore naštejemo vse dadoteke, ki jim v Gitu ne želimo slediti


%VIRI
Navedimo nekaj virov za {\LaTeX}. Ker smo uporabili paket \texttt{hyperref}, lahko na povezave kar kliknete:

\begin{description}
\item[\href{http://www-lp.fmf.uni-lj.si/plestenjak/vaje/latex/lshort.pdf}{\emph{Ne najkrajši uvod v {\LaTeX}}}]
  Ravno pravšnji pregled {\LaTeX}a --- priporočamo!

\item[\url{https://www.sharelatex.com/learn/Main_Page}]
  Naučite se LaTeX v 30 minutah!

\item[\url{http://detexify.kirelabs.org/classify.html}] Če iščete poseben znak, ga tu
  narišete in spletna stran vam pove, kateri ukaz v {\LaTeX}u vam da tak znak.
\end{description}

\end{document}