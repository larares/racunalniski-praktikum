% alt + Z = da imaš vse na istem zaslonu in ne rabiš scrollat do konca vrstice

\documentclass{article} 
% documentclass ima v [] nastavitve in v {} tip dokumenta (v [] smo dali kot primer a4paper in velikost pisave - npr [a4paper, 10pt])

% pogosti tipi dokumentov:
%   article - članek
%   amsart - članki za revije American Mathematical Society
%   beamer - prosojnice za predavanje
%   book - knjiga
%   letter - pismo

% PREAMBULA = ne pišemo nastavitev ampak samo ukaze in nastavitve, ki jih določamo s pomočjo različnih paketov; \usepackage[dodatne nastavitve]{ime paketa} 

\usepackage[utf8]{inputenc} 
% vhodna datoteka je kodirana v sistemu UTF-8

\usepackage[T1]{fontenc}      
% izhodna datoteka je kodirana v sistemu T1


% PAKETI ZA SLOVENŠČINO:
\usepackage[slovene]{babel} 
% za uporabljanje slovenščine in izpisovanje šumnikov
\usepackage{lmodern}
% to je dodaten, načeloma neobvezen paket, ki olepša slovensko pisavo 
% Ostale pogoste izbire pisav:
% \usepackage{times}            % Times New Roman
% \usepackage{palatino}         % Palatino
% \usepackage{concrete}         % Pisava, ki jo je uporabil Donald Knuth v knjigi "Concrete mathematics"


% OSTALI PAKETI:
\usepackage{hyperref}
% paket za povezave znotraj dokumenta in hiper - povezave


% KONEC PREAMBULE

% VSEBINA DOKUMENTA - se vedno začne z:
\begin{document}
% in kasneje konča z \end{document}

% NASLOV IN AVTOR

\title{Zapiski za {\LaTeX}}
% naslov

\author{Lara Reš}
% in avtor

\date{}
% lahko nastavimo še datum, ki ga zapišemo v zavite oklepaje, če pa želimo, da se datum ne izpiše, potem pustimo prazno znotraj oklepajev

\maketitle 
% je ukaz, s katerim dejansko naredimo naslov


% OKOLJA
\begin{abstract}
    To je okolje za povzetek, napišemo kratek opis vsebine članka
\end{abstract}


% Članek lahko hierarhično razdelimo na razdelke (\section), podrazdelke (\subsection), podpodrazdelke (\subsubsection) in podpodpodrazdelke (\paragraph). V večini primerov potrebujemo samo razdelke in podarazdelke.

\end{document}